%\documentclass[paper]{geophysics}
\documentclass[paper,revised]{geophysics}
\usepackage{cleveref} %for cref

% An example of defining macros
\newcommand{\rs}[1]{\mathstrut\mbox{\scriptsize\rm #1}}
\newcommand{\rr}[1]{\mbox{\rm #1}}

\begin{document}

\title{An example \emph{Geophysics} article, \\ with a two-line title}

\renewcommand{\thefootnote}{\fnsymbol{footnote}} 

\ms{GEO-Example} % paper number

\address{
\footnotemark[1]BP UTG, \\
200 Westlake Park Blvd, \\
Houston, TX, 77079 \\
\footnotemark[2]Bureau of Economic Geology, \\
John A. and Katherine G. Jackson School of Geosciences \\
The University of Texas at Austin \\
University Station, Box X \\
Austin, TX 78713-8924}
\author{Joe Dellinger\footnotemark[1] and Sergey Fomel\footnotemark[2]}

\footer{Example}
\lefthead{Dellinger \& Fomel}
\righthead{\emph{Geophysics} example}

\maketitle

\begin{abstract}
  This is an example of using \textsf{geophysics.cls} for writing
  \emph{Geophysics} papers.
\end{abstract}

\section{Introduction}

The nonlinear relation between the physical properties of earth and natural phenomina expalin the wave behaviour is responsible for FWI to stuck in local minima if the starting model is not lie within the basin of attraction. in this both are used together because both have advantages and disadvantages global is best in exploration and local is best in exploitation. PSO is easily parallelized.
Some of the appealing facts of PSO are its convenience, simplicity and easiness of implementation requiring.
The prominent features of PSO are its easy implementation, robustness to control parameters and computation efficiency compared with other existing heuristic algorithms such as genetic algorithm in a continuous problem.
method \ref{method}
\section{Methodology}
\label{method}
This approach involves two steps. First, a coarse velocity model is prepared using PSO by optimizing the depth, interface velocity, and the rate of change of velocity between interfaces. This model serves as an initial model for conventional gradient-based FWI in the second step.
\subsection{Particle Swarm Optimization}
Particle Swarm Optimization (PSO) is a stochastic method developed by James Kennedy and Russell Eberhart in 1995. Inspired by the social behavior of birds flocking to find food, they formulated a mathematical model to simulate this behavior. This model is widely applied to solve various optimization problems. They identified that the fundamental principle guiding birds' food-finding behavior is their ability to communicate with each other. Each bird in the process knows its current position \((x_i(t))\) and best position \((p_i(t))\), determined by evaluating the fitness using a cost function. Additionally, each bird shares its best position with others, contributing to the collective knowledge of the flock's best position \((g(t))\). Each bird's next movement is adjusted by its own best, the flock's best, and its current position. This iterative process continues at each step, ultimately converging towards a globally optimal position through the collaborative effort of all birds. This natural phenomenon is mathematically described by velocity \ref{eqn:pso1}, and position update equation \ref{eqn:pso2}.
\begin{itemize}
	\item \( \mathbf{x}_i(t) \) be the position of particle \( i \) at iteration \( t \).
	\item \( \mathbf{v}_i(t) \) be the velocity of particle \( i \) at iteration \( t \).
	\item \( \mathbf{p}_i(t) \) be the personal best position of particle \( i \) until iteration \( t \).
	\item \( \mathbf{g}(t) \) be the global best position among all particles until iteration \( t \).
	\item \( w \) be the inertia weight.
	\item \( c_1 \) and \( c_2 \) be the cognitive and social acceleration coefficients, respectively.
	\item \( r_1 \) and \( r_2 \) be random numbers uniformly distributed in the range \([0, 1]\).
\end{itemize}

The velocity and position update rules for each particle are given by:
\begin{equation}
	\label{eqn:pso1}
	\mathbf{v}_i(t+1) = w \mathbf{v}_i(t) + c_1 r_1 (\mathbf{p}_i(t) - \mathbf{x}_i(t)) + c_2 r_2 (\mathbf{g}(t) - \mathbf{x}_i(t))	
\end{equation}

\begin{equation}
	\label{eqn:pso2}
	\mathbf{x}_i(t+1) = \mathbf{x}_i(t) + \mathbf{v}_i(t+1)	
\end{equation}

Where:
\begin{itemize}
	\item \( w \) controls the influence of the previous velocity (inertia).
	\item \( c_1 \) and \( c_2 \) represent the trust of the particle in itself and in the swarm, respectively.
	\item \( r_1 \) and \( r_2 \) introduce stochasticity to the particle's movement.
\end{itemize}
Inertia weight \((w)\), acceleration coefficients \((c_1,\) \(c_2)\), population size of swarm, and iterations are the controlling parameters influence the updates. Inertia weight is most important tuning parameter in PSO algorithm which provide a balance between exploration and exploitation abilities of PSO by tuning the contribution of velocity. the value of weight between 0.4 to 0.9 provide a good convergence. \((c_1,\) \(c_2)\) are cognitive and social coefficients decide by the confidence on particle's own and swarms' best position this influence the updated velocity. number of particles in swarm also affect the convergence as search space is increased with number of particles and provide a better convergence, experimentally it was studied that arround 30 particels are good for finding the solution within optimal iterations. number of iterations affects all iterative optimization algorithms same is true for PSO.
\\
This contribution of different parameters for succes of PSO make it is necessary to decide the combination of these parameters precisely. we have performed experiments to find the optimal combination of this values, for these experiment we have choose four nonlinear functions Ackley, Griewank, Rastrigin, and Styblinski-Tang function. this experiment help to choose the optimim value of optimization coeefiecient and finaaly PSO is applied to Schwefel function with parameters choosen frm previous experiment to get parameters which give a best convergence. these experiments as follows.
\begin{enumerate}
<<<<<<< HEAD
	\item Accuracy with parameters: In this experiment optimization is performed for all test functions with different \(c_1\), \(c_2\), and inertia weight \((w)\) as shown in figure \ref{fig:ackley}, \ref{fig:griewank}, \ref{fig:rastrigin}, and \ref{fig:styblinski}. To deal with the inherent randomness of this stochastic method we have perform 50 runs and take the average valus of all as a final optimized values. Accuracy in percentage is calculated using the the equation \ref{eqn:accuracy}.
=======
	\item Accuracy with parameters: In this experiment optimization is performed for all described test functions with different \(c_1\), \(c_2\), and inertia weights as shown in figure \ref{fig:ackley}, \ref{fig:griewank}, \ref{fig:rastrigin}, and \ref{fig:styblinski}. To deal with the inherent randomness of this stochastic method we have perform 50 runs and take the average valus of all as a final optimized values. Accuracy in percentage is calculated using the the equation \ref{eqn:accuracy}.
>>>>>>> 6ed0ddfd9ab075da82508a906152054baf4bc495
	\begin{equation}
		Accuracy (\%) = 100 - \left| \frac{optimal \ value - evaluated \ value}{optimal \ value - maximum \ deviation} \right| \times 100
		\label{eqn:accuracy}
	\end{equation}
Where:
\begin{itemize}
	\item \(Optimal \  value\), best possible value of the objective function.
	\item \(Evaluated \ value\), value of the objective function at a given point in the feasible domain.
	\item \(maximum \ deviation\), largest difference between the optimized value and the actual values.
\end{itemize}
	\begin{sidewaysfigure}	
		\includegraphics[width=\paperwidth]{Fig/ackley_para_vs_accuracy.jpeg}
		\caption{Optimization of the Ackley function with varying inertia weights, \(c_1\), and \(c_2\) coefficients.}
		\label{fig:ackley}
	\end{sidewaysfigure} 
	\begin{sidewaysfigure}	
		\includegraphics[width=\paperwidth]{Fig/griewank_para_vs_accuracy.jpeg}
		\caption{Optimization of the Griewank function with varying inertia weights, \(c_1\), and \(c_2\) coefficients.}
		\label{fig:griewank}
	\end{sidewaysfigure} 
	\begin{sidewaysfigure}	
		\includegraphics[width=\paperwidth]{Fig/rastrigin_para_vs_accuracy.jpeg}
		\caption{Optimization of the Rastrigin function with varying inertia weights, \(c_1\), and \(c_2\) coefficients.}
		\label{fig:rastrigin}
	\end{sidewaysfigure} 
	\begin{sidewaysfigure}	
		\includegraphics[width=\paperwidth]{Fig/styblinski_para_vs_accuracy.jpeg}
		\caption{Optimization of the Styblinski-Tang function with varying inertia weights, \(c_1\), and \(c_2\) coefficients.}
		\label{fig:styblinski}
	\end{sidewaysfigure} 
\end{enumerate}
\section*{Theory}

This is another section. 

\subsection{Equations}

Section headings should be capitalized. Subsection headings should
only have the first letter of the first word capitalized.

Here are examples of equations involving vectors and tensors:
\begin{equation}
\tensor{R} = 
\pmatrix{R_{\rs{XX}} & R_{\rs{YX}} \cr R_{\rs{XY}} & R_{\rs{YY}}} 
=
\tensor{P}_{M\rightarrow R} \; \tensor{D} \; \tensor{P}_{S\rightarrow M}
\;\;\; \tensor{S} \ \ \  ,
\label{SVD}
\end{equation}
and
\begin{equation}
R_{j,m}(\omega) =
\sum_{n=1}^{N} \, \,
P_{j}^{(n)}(\mathbf{x}_R) \, \,
D^{(n)}(\omega) \, \,
P_{m}^{(n)}(\mathbf{x}_S) \ \ \ .
\label{SVDray}
\end{equation}
\ref{SVDray}
Note that the macro for the \verb#\tensor# command has been changed to
force tensors to be bold uppercase, in compliance with current SEG
submission standards. This is so that documents typeset to the old
standards will print out according to the new ones: e.g., tensor
$\tensor{t}$ (note converted to uppercase).

\subsection*{Figures}
\renewcommand{\figdir}{Fig} % figure directory

Figure~\ref{fig:waves} shows what it is about.

\plot{waves}{width=\textwidth}
{This figure is specified in the document by \texttt{
    $\backslash$plot\{waves\}\{width=$\backslash$textwidth\}\{This caption.\}}.
}

\subsubsection{Multiplot} 


The first argument of the \texttt{multiplot} command specifies the
number of plots per row.

\subsection{Tables}


\begin{acknowledgments}
I wish to thank Ivan P\v{s}en\v{c}\'{\i}k and Fr\'ed\'eric Billette
for having names with non-English letters in them.  I wish to thank
\cite{Cerveny} for providing an example of how to make a bib file that
includes an author whose name begins with a non-English character and
\cite{forgues96} for providing both an example of referencing a Ph.D.
thesis and yet more non-English characters.
\end{acknowledgments}

\append{Appendix example}
\label{example}

According to the new SEG standard, appendices come before references.

\begin{equation}
\frac{\partial U}{\partial z} = 
\left\{
  \sqrt{\frac{1}{v^2} - \left[\frac{\partial t}{\partial g}\right]^2} +
  \sqrt{\frac{1}{v^2} - \left[\frac{\partial t}{\partial s}\right]^2}
\right\}
\frac{\partial U}{\partial t}
\label{eqn:partial}
\end{equation}
It is important to get equation~\ref{eqn:partial} right. See also
Appendix~\ref{equations}.

\append[equations]{Another appendix}

\begin{equation}
\frac{\partial U}{\partial z} = 
\left\{
  \sqrt{\frac{1}{v^2} - \left[\frac{\partial t}{\partial g}\right]^2} +
  \sqrt{\frac{1}{v^2} - \left[\frac{\partial t}{\partial s}\right]^2}
\right\}
\frac{\partial U}{\partial t}
\label{eqn:partial2}
\end{equation}
Too lazy to type a different equation but note the numeration.

The error comparison is provided in Figure~\ref{fig:errgrp}.

\sideplot{errgrp}{width=0.8\textwidth}
{This figure is specified in the document by \texttt{
    $\backslash$sideplot\{errgrp\}\{width=0.8$\backslash$text\-width\}\{This caption.\}}.
}

\append{The source of this document}

\verbatiminput{geophysics_paper.tex}

\append{The source of the bibliography}

\verbatiminput{geophysics_reference.bib}

\newpage

\bibliographystyle{seg}  % style file is seg.bst
\bibliography{geophysics_reference}

\end{document}
