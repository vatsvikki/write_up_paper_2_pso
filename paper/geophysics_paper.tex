%\documentclass[paper]{geophysics}
\documentclass[paper,revised]{geophysics}
\usepackage{cleveref} %for cref

% An example of defining macros
\newcommand{\rs}[1]{\mathstrut\mbox{\scriptsize\rm #1}}
\newcommand{\rr}[1]{\mbox{\rm #1}}

\begin{document}

\title{An example \emph{Geophysics} article, \\ with a two-line title}

\renewcommand{\thefootnote}{\fnsymbol{footnote}} 

\ms{GEO-Example} % paper number

\address{
\footnotemark[1]BP UTG, \\
200 Westlake Park Blvd, \\
Houston, TX, 77079 \\
\footnotemark[2]Bureau of Economic Geology, \\
John A. and Katherine G. Jackson School of Geosciences \\
The University of Texas at Austin \\
University Station, Box X \\
Austin, TX 78713-8924}
\author{Joe Dellinger\footnotemark[1] and Sergey Fomel\footnotemark[2]}

\footer{Example}
\lefthead{Dellinger \& Fomel}
\righthead{\emph{Geophysics} example}

\maketitle

\begin{abstract}
  This is an example of using \textsf{geophysics.cls} for writing
  \emph{Geophysics} papers.
\end{abstract}

\section{Introduction}

The nonlinear relation between the physical properties of earth and natural phenomina expalin the wave behaviour is responsible for FWI to stuck in local minima if the starting model is not lie within the basin of attraction. in this both are used together because both have advantages and disadvantages global is best in exploration and local is best in exploitation. PSO is easily parallelized.
method \ref{method}
\section{Methodology}
\label{method}
This approach involves two steps. First, a coarse velocity model is prepared using PSO by optimizing the depth, interface velocity, and the rate of change of velocity between interfaces. This model serves as an initial model for conventional gradient-based FWI in the second step.
\subsection{Particle Swarm Optimization}
Particle swarm optimization is a stochastic method was formulated by James Kennedy and Russel Eberhart in 1995 when both are involving in understanding the social behaviour of folk birds used for locating food finally this conclude to a mathematical expression a by simulating the behavior of birds further this is used to solve optimization problems. They found the basic fundamental on which birds food finding work is their ability of communication with each other. In this process every bird knows their current \(x_i(t)\), and best position \(p_i(t)\) by computing the fitness value using cost function, every bird also know the best position  \(g(t)\) of folk by sharing their own best positions. The movement of every bird for next step is guided by their own, folk's best position, and by the velocity with which particle is travelling. This process is repeated at every step and finally a globaly best position is achieved by the collective work of all birds. The mathematical expression of this natural phenomina is explained in equation \ref{eqn:pso1}, \ref{eqn:pso2}.
\begin{itemize}
	\item \( \mathbf{x}_i(t) \) be the position of particle \( i \) at iteration \( t \).
	\item \( \mathbf{v}_i(t) \) be the velocity of particle \( i \) at iteration \( t \).
	\item \( \mathbf{p}_i(t) \) be the personal best position of particle \( i \) until iteration \( t \).
	\item \( \mathbf{g}(t) \) be the global best position among all particles until iteration \( t \).
	\item \( w \) be the inertia weight.
	\item \( c_1 \) and \( c_2 \) be the cognitive and social acceleration coefficients, respectively.
	\item \( r_1 \) and \( r_2 \) be random numbers uniformly distributed in the range \([0, 1]\).
\end{itemize}

The velocity and position update rules for each particle are given by:
\begin{equation}
	\label{eqn:pso1}
	\mathbf{v}_i(t+1) = w \mathbf{v}_i(t) + c_1 r_1 (\mathbf{p}_i(t) - \mathbf{x}_i(t)) + c_2 r_2 (\mathbf{g}(t) - \mathbf{x}_i(t))	
\end{equation}

\begin{equation}
	\label{eqn:pso2}
	\mathbf{x}_i(t+1) = \mathbf{x}_i(t) + \mathbf{v}_i(t+1)	
\end{equation}

Where:
\begin{itemize}
	\item \( w \) controls the influence of the previous velocity (inertia).
	\item \( c_1 \) and \( c_2 \) represent the trust of the particle in itself and in the swarm, respectively.
	\item \( r_1 \) and \( r_2 \) introduce stochasticity to the particle's movement.
\end{itemize}
\section*{Theory}

This is another section. 

\subsection{Equations}

Section headings should be capitalized. Subsection headings should
only have the first letter of the first word capitalized.

Here are examples of equations involving vectors and tensors:
\begin{equation}
\tensor{R} = 
\pmatrix{R_{\rs{XX}} & R_{\rs{YX}} \cr R_{\rs{XY}} & R_{\rs{YY}}} 
=
\tensor{P}_{M\rightarrow R} \; \tensor{D} \; \tensor{P}_{S\rightarrow M}
\;\;\; \tensor{S} \ \ \  ,
\label{SVD}
\end{equation}
and
\begin{equation}
R_{j,m}(\omega) =
\sum_{n=1}^{N} \, \,
P_{j}^{(n)}(\mathbf{x}_R) \, \,
D^{(n)}(\omega) \, \,
P_{m}^{(n)}(\mathbf{x}_S) \ \ \ .
\label{SVDray}
\end{equation}
\ref{SVDray}
Note that the macro for the \verb#\tensor# command has been changed to
force tensors to be bold uppercase, in compliance with current SEG
submission standards. This is so that documents typeset to the old
standards will print out according to the new ones: e.g., tensor
$\tensor{t}$ (note converted to uppercase).

\subsection*{Figures}
\renewcommand{\figdir}{Fig} % figure directory

Figure~\ref{fig:waves} shows what it is about.

\plot{waves}{width=\textwidth}
{This figure is specified in the document by \texttt{
    $\backslash$plot\{waves\}\{width=$\backslash$textwidth\}\{This caption.\}}.
}

\subsubsection{Multiplot} 


The first argument of the \texttt{multiplot} command specifies the
number of plots per row.

\subsection{Tables}


\begin{acknowledgments}
I wish to thank Ivan P\v{s}en\v{c}\'{\i}k and Fr\'ed\'eric Billette
for having names with non-English letters in them.  I wish to thank
\cite{Cerveny} for providing an example of how to make a bib file that
includes an author whose name begins with a non-English character and
\cite{forgues96} for providing both an example of referencing a Ph.D.
thesis and yet more non-English characters.
\end{acknowledgments}

\append{Appendix example}
\label{example}

According to the new SEG standard, appendices come before references.

\begin{equation}
\frac{\partial U}{\partial z} = 
\left\{
  \sqrt{\frac{1}{v^2} - \left[\frac{\partial t}{\partial g}\right]^2} +
  \sqrt{\frac{1}{v^2} - \left[\frac{\partial t}{\partial s}\right]^2}
\right\}
\frac{\partial U}{\partial t}
\label{eqn:partial}
\end{equation}
It is important to get equation~\ref{eqn:partial} right. See also
Appendix~\ref{equations}.

\append[equations]{Another appendix}

\begin{equation}
\frac{\partial U}{\partial z} = 
\left\{
  \sqrt{\frac{1}{v^2} - \left[\frac{\partial t}{\partial g}\right]^2} +
  \sqrt{\frac{1}{v^2} - \left[\frac{\partial t}{\partial s}\right]^2}
\right\}
\frac{\partial U}{\partial t}
\label{eqn:partial2}
\end{equation}
Too lazy to type a different equation but note the numeration.

The error comparison is provided in Figure~\ref{fig:errgrp}.

\sideplot{errgrp}{width=0.8\textwidth}
{This figure is specified in the document by \texttt{
    $\backslash$sideplot\{errgrp\}\{width=0.8$\backslash$text\-width\}\{This caption.\}}.
}

\append{The source of this document}

\verbatiminput{geophysics_paper.tex}

\append{The source of the bibliography}

\verbatiminput{geophysics_reference.bib}

\newpage

\bibliographystyle{seg}  % style file is seg.bst
\bibliography{geophysics_reference}

\end{document}
